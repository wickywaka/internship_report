% For help on subfiles see https://www.sharelatex.com/learn/Multi-file_LaTeX_projects
\documentclass[../main.tex]{subfile}

\begin{document}
		
		\paragraph{} CuckooDroid is an extension of Cuckoo Sandbox the Open Source software for automating analysis of suspicious files. CuckooDroid brigs to cuckoo the capabilities of execution and analysis of android application \cite{cuckoodroid_docs}. For more information about the cuckoo sandbox, readers are encouraged to visit the cuckoo sandbox website \cite{cuckoo_website}. Currently, CuckooDroid only supports Android 4.1.
		
		\paragraph{} CuckooDroid can be downloaded from the CuckooDroid github repository \cite{cuckoodroid_github} by following the guidelines provided there. For more step-by-step installation guide, readers are encouraged to have a look at CuckooDroid documentation \cite{cuckoodroid_docs}. Because of changes in android emulator (goldfish) and android SDK the CuckooDroid documentation are not precisely accurate and some deviations are required from it in order to make the CuckooDroid work. By following the CuckooDroid documentation with a few changes discussed later in this chapter, it should be fairly easy for reader to configure his CuckooDroid setup. 

		
		\section{CuckooDroid architecture}
		\paragraph{} In this section we will describe the architecture of CuckooDroid. There are main two parts a "Host" and a "Guest". Below is an excerpt from the CuckooDroid documentation \cite{cuckoodroid_docs}:
		
		\say{This documentation refers to Host as the underlying operating systems on which you are running Cuckoo (generally being a GNU/Linux distribution) and to Guest as the Windows virtual machine used to run the isolated analysis.}
		
		\paragraph{} We will be configuring CuckooDroid with Android Emulator (Goldfish) and figure \ref{fig:cuckoodroid_avd_arch} shows the architecture CuckooDroid with Android Emulator. As it can be seen in the figure \ref{fig:cuckoodroid_avd_arch} that there are two main parts, Cuckoo Sandbox and Android Emulator.
		
		\begin{figure}
			\centering
			\fbox{\includegraphics[scale=0.4]{android_avd_arch.png}}
			\caption{CuckooDroid architecture with AVD}
			\label{fig:cuckoodroid_avd_arch}
		\end{figure}
		
		\paragraph{} Cuckoo Sandbox is responsible for managing the android emulator and generating report at the end of analysis. Android Emulator executes the application, gather some information from it and reports it back to Cuckoo Sandbox. Below is the description of some of the main parts shown in figure \ref{fig:cuckoodroid_avd_arch}
		\begin{itemize}
			\item \textbf{Python Agent} Executed on AVD and is responsible for receiving APK file, Analysis code, configuration and executing analysis. It also provides constant status updates to Host.
			\item \textbf{Python Analyzer} Android analyzer component that is sent to the guest machine at the beginning of the analysis. This is the main part that executes application, send dropped files back to host, send screenshots back to host, interact with the application if required. It is also responsible for ending the analysis and sending back some log files back to host. It has modular structure and new modules can be added very easily.
			\item \textbf{Xposed Framework} A framework for modules that can change the behavior of the system and apps without affecting any APKs. The version in use only work up to Android 4.1.2 (API 16). In CuckooDroid two modules are used with this Framework:
			\begin{enumerate}

				\item \textbf{Droidmon:} Dalvik  API Call monitoring module, it hooks up API calls and prints it into logcat, anaylsis code takes it form logcat and store it in a log file which is sent back to host at the end of anaylsis.
				\item \textbf{Emulator Anti-Detection:} Implements some Emulator Anti-detection techniques. Its not specified anywhere as to exactly which techniques does it use. For more details on this topic see Chapter \ref{sec:Chp4}.
			\end{enumerate}
		\end{itemize}
		\paragraph{} In the rest of this chapter we will walk through configuring CuckooDroid and some of the changes which are necessary to make it work with latest Android Emulator and latest Android versions.
		\section{CuckooDroid Installation}\label{sec:cuckoodroid_installation}
		\paragraph{} Installation of CuckooDroid is pretty straight forward and can be easily done by following the Easy integration script available on the CuckooDroid github repository \cite{cuckoodroid_github}. For the configuration and requirements follow the instruction in the CuckooDroid documentations \cite{cuckoodroid_docs}. In the next two paragraphs we will give a bit more intuitive description of how CuckooDroid works which will help in understanding some of steps in installation and configuration step. Please note that there are some changes which are required to make CuckooDroid work and those are discussed in section \ref{sec:cuckoodroid_patching}.
		\paragraph{} The basic components of CuckooDroid can be seen in figure \ref{fig:cuckoodroid_avd_arch}. There are two basic parts of CuckooDroid host and guest. We run the cuckoo sandbox on host which handles the emulator. In order to do analysis our guest needs to be a rooted AVD with Xposed Framework \cite{xposed_module_repo} two of its modules Droidmon and Emulator Anti-Detection. We also need Python 2.7 to run the python agent and analyzer code on our guest.
		\paragraph{} Once we run the Cuckoo sandbox and submit a sample to it, it makes a new copy of our specified AVD and launches it. Once the machine boots, the cuckoo sandbox runs the python agent using Android Debugger (adb) and sends the APK file, configuration file, and analysis code to it and orders it to start executing the analysis code. The analysis code uses the apk module to install the sample and executes it. The DroidMon module of xposed Framework monitors the specified function calls using hooks and prints it on the logcat. Screenshot modules takes screenshots of the android screen and sends it to host. Once the analysis is completed, the analyzer sends all the required files back to host and terminates. Then host kills the emulator and starts making a report out of these files.
		\section{CuckooDroid required patching for Android 4.1}\label{sec:cuckoodroid_patching}
		\paragraph{} Initially, we were expecting that CuckooDroid will work out of the box but that wasn't the case. In this section we will talk about some of the changes which were required to make CuckooDroid work with Android 4.1. There are two main steps, one is the persistent root problem and the second one is some bug fixes in python analyzer code.
		
		\subsection{Android 4.1 Persistent Rooting}
		\paragraph{} Now coming towards the first fix which was that the root wasn't staying persistent across reboots. It was the result of changes in Android emulator. More details and solutions about this problem can be found in issue 4 of CuckooDroid github repository \cite{cuckoodroid_root_issue}. The solution can be found in the same issue, according to which we needed to copy the system.img file from the android SDK directory to the AVD directory. The following command can be used to do that, make sure you replace the paths according to the location of your android SDK and AVD.
		
		\begin{lstlisting}[numbers=none]
			$ cp ~/Android/Sdk/system-images/android-16/default/armeabi-v7a/system.img ~/.android/avd/aosx.avd/system-qemu.img
		\end{lstlisting}
		
		\paragraph{} After the above step, we need to start the emulator with command given below:
		\begin{lstlisting}[numbers=none]
			$ emulator -avd aosx -qemu -nand -system,size=0x1f400000,file=~/.android/avd/aosx.avd/system-qemu.img&
		\end{lstlisting}
		
		The follow the standard steps of rooting described in cuckoodroid documentations \cite{cuckoodroid_docs}. I also made some youtube tutorials to help the readers with the process. You can find the related video tutorial here \cite{rooting_youtube}.
		
				
		\subsection{Bug fixes in Analyzer code}
		\paragraph{} The second fix is in the python code that is executed on Android. The CuckooDroid analyzer and agent code heavily relies on Subprocess Management library \cite{subprocess_management_library} (specifically subprocess.Popen() method). During debugging we noticed that Most of the calls to subprocess.Popen() method never returns resulting in analysis critical time out. Upon further research we found an issue relating to a similar problem in python 3.7 \cite{subprocess.popen_issue16255}, which seems to be the case here. The subprocess.Popen function uses /bin/sh in Unix environments. Android is detected as a Unix environment, but has moved that executable to "/system/bin/sh". This can be worked around by adding a parameter executable="/system/bin/sh" to all the subprocess.Popen calls. One related problem was that in some parts of the analyzer, like in adb.py os.popen was used to execute shell commands. This function was also failing to return so we have to replace it with subprocess.Popen. The related video tutorial can be found here \cite{subprocess_youtube}
		
		\paragraph{} The next limitation that we faced was that CuckooDroid only supports Android upto 4.1, which is quiet older and in order to analyze applications that are targeting higher version of androids we need to find a way to run CuckooDroid on higher version of Androids. In the next sections we will describe the steps and results of our attempts to upgrade to Android 5.1 and Android 7.0.

		\section{CuckooDroid with Android 5.1}
		\paragraph{} Just like with Android 4.1, here we again faced the persistent root problem. Another problem was that we couldn't use the python binaries for android that comes with CuckooDroid because from this version Android make it a requirement for binaries to support PIE mode. In the following subsections we will describe in more detail about how did we solved these problems. Here we used the x86 image just to save time during testing. This rooting procedure works on arm image too but that speed of arm image is very slow and was wasting a lot of time during the development and debugging.
		
		\subsection{Android 5.1 AVD persistent rooting procedure} \label{sec:android_5.1_root}
		\paragraph{} The rooting procedure for Android 5.1 is a bit different than that of Android 4.1 in the sense that we needed updated version of tools like SuperSU, XposedFramework etc. Below are the required steps to root Android 5.1 AVD and keep the root persistent.
		\begin{enumerate}
			\item First copy the system.img to your avd directory and rename it as system-qemu.img: For example for Nexus One API 22 x86: \newline 
			\begin{lstlisting}[language=bash, numbers=none]
			cp ~/Android/Sdk/system-images/android-22/default/x86/system.img ~/.android/avd/Nexus_One_API_22_x86.avd/system-qemu.img
			\end{lstlisting}
						
			
			\item Then start the AVD Machine with \newline 
			\begin{lstlisting}[language=bash, numbers=none]
			emulator @Nexus_One_API_22_x86 -verbose -writable-system
			\end{lstlisting}
						
			
			\item Run the script below which was made with the help of stackexchange answer by a user with the name xavier\textunderscore fakerat \cite{android_emulator_7.1_root_stackexchange}.
			\begin{lstlisting}[language=bash]
			#!/usr/bin/env bash
			# Inspired by: https://android.stackexchange.com/questions/171442/root-android-virtual-device-with-android-7-1-1/176447
			#this script is meant for easy creation on an analysis machine for android emulator avd
			
			#Path to the local installation of the adb - android debug bridge utility.
			#ADB=/home/wra/Andriod/Sdk/platform-tools/adb
			ADB=$(which adb)
			if [ ! -f $ADB ]
			then
			echo -e "\n-Error: adb path is not valid."
			exit
			fi
			echo -e "\n-adb has been found."
			# First we root the emulator using Superuser
			echo -e "\n-Pushing /system/xbin/su binary"
			$ADB root
			$ADB remount
			# one of these is correct, we do both just to be on safe side
			$ADB push nougat/SuperSU-v2.82-201705271822/x86/su.pie /system/xbin/su
			$ADB shell chmod 06755 /system/xbin/su
			$ADB push nougat/SuperSU-v2.82-201705271822/x86/su.pie /system/bin/su
			$ADB shell chmod 06755 /system/bin/su
			echo -e "\n-Installing application Superuser"
			$ADB install nougat/eu.chainfire.supersu_2.82.apk
			# Install Xposed Application
			echo -e "\n-Installing Xposed Application"
			$ADB install nougat/XposedInstaller_3.1.4.apk
			# Install Droidmon Application
			echo -e "\n-Installing Droidmon Application"
			$ADB install hooking/Droidmon.apk
			# Install Termux application
			$ADB install nougat/com.termux.apk
			# Install Anti Emulator Detection Application
			echo -e "\n-Installing Anti Emulator Detection Application"
			$ADB install hooking/EmulatorAntiDetect.apk
			$ADB push anti-vm/fake-build.prop /data/local/tmp/
			$ADB push anti-vm/fake-cpuinfo /data/local/tmp/
			$ADB push anti-vm/fake-drivers /data/local/tmp/
			# Install Content Generator
			echo -e "\n-Installing Content Generator"
			$ADB install apps/ImportContacts.apk
			# Install Cuckoo Agent and Python for ARM
			echo -e "\n-Pushing Agent executing scripts"
			$ADB push ../../agent/android/python_agent/. /data/local/ 
			$ADB shell chmod 06755 /data/local/aapt
			$ADB shell chmod 06755 /data/local/get_terminal.sh
			$ADB shell chmod 06755 /data/local/run_agent.sh
			echo -e "\n***** Run the following commands and you are good to go*****\n"
			echo "adb shell"
			echo "cd /system/bin/"
			echo "su root"
			echo "su --install"
			echo "su --daemon&"
			echo "setenforce 0"
			echo -e "\n"
			echo -e "*NOTE"
			echo " To make sure everything is working fine after the rooting is done, do the following:"
			echo " Open android emualtor and run termux. Then type python2 to make sure to that python is installed"
			echo " Then open adb shell Run the get_terminal.sh script and then type python2 to verify that you "
			echo " that you can run python with root previledges. After that run run_agent.sh and verify that agent is running."
			\end{lstlisting}
			
		\item Run the SuperSu app and if the device is rooted, it will ask for update, update the app, don't reboot yet.
		\item After update, open Xposedframework app and click install there, it will ask for administrator permission, click grant.
		\item Enable Droidmon and anti-emulator modules.
		\item Soft reboot the machine, (It will be somewhere in xposed app)
		\item After reboot verify root by opening SuperSU app and it doesn't give you an error and show xposedframework app, the device is rooted.
		\item Close the machine, (The emulator 27.0.1 will save the snapshot of the machine and will load it at next start)
		\item start the machine with -writable-system option (without it the loaded machine is not rooted)and it will automatically load the previously saved snapshot, that means it will be up and running in no time then verify root access by opening SuperSu app or by running another app that required root access, then close the machine.
		\item start the machine again with adding one more option to above command, that is -no-snapshot, this time the machine will boot properly and verify the root by running SuperSU app or another app that needs root.
					
		\end{enumerate}
		\subsection{Python 2.7 on Android 5.1} \label{sec:android_5.1_python}
		\paragraph{} In order to do the analysis we need to have python 2.7 in our Android 5.1 guest. CuckooDroid comes with python 2.7 binaries which are compiled for Android 4.1, which doesn't work with Android 5.1 because with the release of Android 5.1 Android requires all dynamically linked executables to support PIE (position-independent executables). So we either need python compiled with PIE support. We can either compile it ourselves or we can use pre-compiled python with PIE support. Compiling python from source for Android was harder than we thought and we left it in favor of the pre-compiled solution because of simplicity. 
		
		\paragraph{Pre-Compiled python for Android 5.1+}
		\paragraph{} Now, as we said above we used the pre-compiled python 2.7 that can be installed inside the Termux application \cite{termux_website}. Termux is an Android terminal emulator and Linux environment application that provides several linux command line tools on android making it really powerful. The app doesn't require root permissions which means it can't access or modify files outside of application. We needed a workaround so that we can run termux shell as root and have access to all system. Below are steps required to achieve that:	
				
		\begin{enumerate}
		\item Download Termux apk from google playstore or apkmirror and install it to a rooted emulator
		\item Open the installed app and type "pkg search python"
		\item Install python2 using pkg install python2 (take help from step2 to determine exact command)
		\item Now type python2 and you will have a working python. But this only works inside the app and can't be run as sudo/root.
		\item To be able to run python(that comes with termux) as root from adb, we will need to download the script \cite{tsu_cswl}  (A fork of this which support optional command is here \cite{tsu_kiney}). Tsu is an su wrapper for the terminal emulator, Termux \cite{termux_website}.
		\item Push this script to your emulator and run it with -e option.
		\item It will drop you into root shell from termux.
		\item Now we need to add the code inside the script so that instead of launching a shell it run our agent.py script.
		\item To do that go to line 100 which is:
		\begin{lstlisting}[language=bash, numbers=none]
		exec "$s" --preserve-environment -c "LD_LIBRARY_PATH=$LD_LIBRARY_PATH $ROOT_SHELL" 
		\end{lstlisting}
		Step 10: Edit the line as follows:
		\begin{lstlisting}[language=bash, numbers=none]
		exec "$s" --preserve-environment -c "LD_LIBRARY_PATH=$LD_LIBRARY_PATH python2 /data/local/agent.py" 
		\end{lstlisting}
		
		\item Now run this script with "-e" and you will see that our agent.py script is been fired. We will save this script as agent.sh and replace our old agent.sh by it.
		\item Now we need to add "-e" option to the part of code which fires up agent.sh, it can be located in the start\textunderscore agent() method of "cuckoo-droid/modules/machinery/avd.py"
		\item Due to some reasons we ran into some permissions problems like not being able to read hooks.json file. In logcat droidmon was complaining about not having permission to read hooks.json. This can be fixed by setting the global read flag on hooks.json after it is copied in cuckoodroid/analyzer/android/analyzer.py Below is the snippet of code:
		\begin{lstlisting}[language=python]
		def prepare(self):
			# Initialize logging.
			init_logging()
			
			# Parse the analysis configuration file generated by the agent.
			self.config = Config(cfg="analysis.conf")
			
			# We update the target according to its category. If it's a file, then
			# we store the path.
			if self.config.category == "file":
			self.target = os.path.join("/data/local/tmp", str(self.config.file_name))
			shutil.copyfile("config/hooks.json", "/data/local/tmp/hooks.json")
			os.chmod("/data/local/tmp/hooks.json", 0754) # Give set global read flag on hooks.json
			# If it's a URL, well.. we store the URL.
			else:
			self.target = self.config.target
		\end{lstlisting}
		
		\item Another problem was in the screenshot module, it wasn't able to read screenshots and this can be solved by using the same method as above i.e, setting global read at the place where the file is created and also changing the location of file, which in this case was because of sdcard being read-only and this function wasn't able to save the image there. Below is the modified code of take\textunderscore screenshot() method in adb file:
		\begin{lstlisting}[language=python]
			def take_screenshot(filename):
				proc1= subprocess.Popen("/system/bin/screencap -p /data/local/screenshots/"+filename, # "/data/local/screenshots/" Fix for /sdcard read only
				stdout=subprocess.PIPE,
				shell=True,
				stderr=subprocess.PIPE,
				executable='/system/bin/sh')
				stdout, stderr = proc1.communicate()
				
				if len(stdout)>0:
				log.info("take_screenshot stdout: %s", stdout)
				if len(stderr)>0:
				log.info("take_screenshot stderr: %s", stderr)
				
				os.chmod("/data/local/screenshots/"+filename, 0744)
				return "/data/local/screenshots/"+filename
		\end{lstlisting}
		
		\end{enumerate}
		
		\section{CuckooDroid with Android 7.0}
		\paragraph{} We didn't have much success with Android 7.0. We could't pass the persistent rooting problem and the machine wouldn't boot after the rooting procedure as described for Android 5.1. Furthermore it was extremely slow and was almost unusable so we decided to move on to more pressing matters.
		
		\section{Improvements in CuckooDroid and pull request}
		\paragraph{} Once we had the cuckoodroid working, we decided to add some features to it. We wrote one additional module for CuckooDroid analyzer. The goal was to improve code coverage by interacting with app in intelligent manner and then to introduce some random events using Monkey \cite{monkey_tool} tool. Monkey tool is described as "The Monkey is a program that runs on your emulator or device and generates pseudo-random streams of user events such as clicks, touches, or gestures, as well as a number of system-level events. You can use the Monkey to stress-test applications that you are developing, in a random yet repeatable manner" \cite{monkey_tool}. Our module is called IntentGun and the meaning of this name will become apparent in the next paragraph.
		\paragraph{} As we said in chapter \ref{sec:intro} that an android application have different entry points called activities, receivers and services and that these can be started using intents. So in this module we make a list of all activities, receivers and services contained inside an app and then use the activity manager utility \cite{am} to start these activities, receivers and services. The commands are give below:
		
		\begin{lstlisting}[language=bash]
			# For Activity
			$ am start -n Package/Activity -a action -c category data
			
			# For receivers
			$ am broadcast -n Package/Receiver -a action -c category
			
			# For servicies
			$ am start -n Package/Activity -a action -c category data
		\end{lstlisting}
		\paragraph{} This module is still work in progress as the data section is not easy because only Multipurpose Internet Mail Extensions (MIME) type is given in the Manifest.xml file and we need to provide data as input to this command and on top of that sometimes these MIMEs are custom defined by the app developers. Once we are done with executing all entry points to an application, then we run the Monkey to generate random events. We used the following command to run the Monkey:
		\begin{lstlisting}[language=bash, numbers=none]
			$ monkey --ignore-crashes --ignore-security-exceptions --monitor-native-crashes -p package --throttle 500 500
		\end{lstlisting}
		
		\paragraph{} Its important to note here that this tool is used to stress test applications and we couldn't manage to make it work in stable condition. Usually it crashes the emulator altogether \cite{issue52}.
		
		\paragraph{} We also upgraded Cuckoo-Droid to support latest androgaurd and made a pull request to the Cuckoo-Droid official repository\cite{cuckoodroid_github}. We decided not to include the intent gun module because of its unstable nature. The pull request can be found here \cite{cuckoodroid_pullreq}. Due to the Cuckoo-Droid maintainers being inactive, we also decided to keep a mirror of the update version at QuoScient github with updated documentation so that people who want to use it can see the updated documentation \cite{cuckoodroid_quoscient}. With this pull request we hope to make cuckoodroid stable enough that it works right out of the box. We were also continuously in touch with the Cuckoo Sandbox team.
		
		\section{Chapter Conclusion}
		\paragraph{} In this chapter we explained how CuckooDroid works, which kind of changes did we need to make in order to make it work and what improvements did me make in it. After doing the work described, in this chapter we also reached to following conclusions:
		\paragraph{Emulator Anti Detection} Although during our dynamic analysis we didn't analyzed enough malicious samples to make an estimate of the usage of emulation detection techniques in these samples, but a little search on the Internet revealed several techniques that were very easy to implement and integrate into an application and thus it can be very easy for a determined threat actor to add them to their code and can make its way to popular android application stores like google play store easily or can stay undetected in general. We will further explore this topic in the chapter \ref{sec:Chp4}.
		
		\paragraph{Support for higher version android} We also reached a conclusion that there is dire need for android malware analysis paltforms that run the latest Android because only 1.5\% users are using Android 4.1 \cite{distribution_dashboard} and this number is declining, also more and more apps are dropping support for android 4.1. For future development of cuckoodroid we think Frida, a dynamic instrumentation framework is a best candidate for instrumentation, because it is well maintained, support latest android operating systems, can be modified very quickly and is cross-platform. We will introduce Frida in Chapter \ref{sec:Chp5}. For a more quick solution, we recommend to compile python with PIE support or get it the way we did it i.e from Termux.
		
		\paragraph{Slow emulator performance} With the android emulators running higher versions of Android, there is big performance penalty. Running an arm image of android on x86 based PC system is also one of the reason of performance degradation. One solution to the problem can be using Houdini translation that allows android apps written for arm processors to be executed on x86 android image. We didn't investigated this solutions further and can be done as a separate project.
		
		\paragraph{Native Code} Currently CuckooDroid only detects if native code is being used in an application. It would be nice to have the capabilities of decompiling/disassembling of native code and include that into analysis results. In Chapter \ref{sec:Chp5}, We will talk about a tool called Frida, which supports hooking native code on Android and be xposed framework can be replaced by it.

\end{document}